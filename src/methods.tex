\section{Image Processing Methods}

All MRI volumes will be processed using Advanced Normalization Tools
(ANTs) \href{http://www.picsl.upenn.edu/ANTS}{(click here for
link)}\cite{Avants2011a,Avants2011,Murphy2011,Tustison2010,Tustison2011,Tustison2011a}
via previously reported procedures validated as highly accurate
\cite{Klein2009,Klein2010,Murphy2011,Avants2011}.  In general, ANTs-based
processing for multiple modalities is based on, first, defining a
template to guide the population normalization and segmentation
procedures \cite{Avants2011a,Avants2011}.  Traditionally, this
approach was used for single modalities but, more recently, is
extended for multiple modalities via multiple-modality cohort-specific templates
that capture the average shape and appearance of T1, DTI and
functional images, as in \cite{Kim2010,Avants2011a,Jain2012,Tustison2012}.  We use the
template to guide brain extraction for all modalities and subsequent
tissue or neuroanatomical parcellation \cite{Avants2011a,Wang2013}.  

Our prior work on TBI reveals effects in thalamus \cite{Avants2008a}
and thalamocortical networks \cite{J.T.Duda2008}.  We used ANTs
methods to uncover widespread structural effects cross-sectionally
\cite{Kim2008} and (more preliminarily) longitudinally \cite{Kim2013}.
In terms of function, we identified TBI effects in default mode
network \cite{Duda2013} and cerebral blood flow \cite{Kim2010}.  Thus,
multiple modality procedures enable several complementary measurements
to characterize TBI:

\noindent{\em Jacobian-based Volume from T1-weighted MRI:} Image
registration methods are able to use T1 MRI to obtain voxel-wise
measurement of the subject structural volume relative to a group local
template or relative to a baseline timepoint.  These measurements are
used in Tensor-Based Morphometry (TBM) which, when performed
symmetrically \cite{Yushkevich2010a,Das2012}, is highly accurate and
provides detailed information on shape change over time
\cite{Brambati2007,Hua2011,Kim2013} or across a population
\cite{Kim2008,Massimo2009,Morgan2011,Hanson2010,Hanson2012}.  

\noindent{\em Cortical thickness from T1-weighted MRI:}  A measurement
that is both sensitive and specific in neurodegeneration
\cite{Stricker2012,Libon2012,McMillan2013} and which is associated
with cognitive decline \cite{Dickerson2011,Avants2010}.  Cortical
thickness has yet to be applied to mTBI but we believe that it will
explain variance in recovery potential within the mTBI population in
particular in the anterior efferent projections of thalamocortical networks
such as the middle frontal and orbitofrontal cortex.

\noindent{\em Measurements derived from DTI:} Fractional anisotropy
measurement of white matter integrity that reveal connectivity
components of neurodegeneration \cite{Zhang2009}, aging
\cite{Kochunov2011} potentially captures a different aspect of the
disease process \cite{Englund2004} and may differentiate tau and
TDP-43 effects \cite{McMillan2013b}.  Relatedly, mean, radial or axial
diffusion relate to age and cognition
\cite{Wu2011,Pal2011,Bava2011,ODwyer2012,Bjoernebekk2012},
neurodegeneration \cite{Salami2012,Whitwell2010} and are particularly
sensitive to TBI \cite{Huang2009,Fox2013}.  We will characterize mean and radial diffusion
within thalamocortical networks.

\noindent{\em BOLD or ASL Network Analysis:} BOLD-based network analysis
\cite{Spoormaker2010,Sanz-Arigita2010} has emerged as a powerful tool
with specific value in both brain injury
\cite{Mayer2011,Scheibel2012,Zhou2013}, the study of intervention
strategies \cite{Roy2010,FeldsteinEwing2011} and measurement of pain
\cite{Mayhew2013}.  ASL may also be employed in this manner
\cite{Jann2013} and has been processed with ANTs tools and canonical
correlation analysis \cite{Duda2013}.  We propose network measures
based on constant graph density across subjects
which controls for inter-subject variation in base correlation levels
\cite{Liu2008, Power2011, Schwarz2011, Braun2012, Liang2012}. 
On the binarized graphs, we compute betweenness, small-worldness and
related metrics. 

\noindent{\em Cerebral blood flow (CBF) from ASL:}  This functional
quantitative measure (versus the relative values provided by BOLD) has
the potential to reveal alterations in the brain due to injury
\cite{Kim2008}, pain \cite{Howard2011}, pharmacological intervention \cite{Black2010,Jenkins2012} or that precede visible structural change and may
indicate cortical reorganization \cite{Hayward2010}.  ASL-based CBF
also accurately recapitulates PET imaging in Alzheimer's disease \cite{Chen2011,Mak2012}.
CBF is a more repeatable functional measurement than BOLD
\cite{Liu2007,Aguirre2012}, may be used in network analysis in lieu of
or combination with 
BOLD \cite{Duda2013} and provides a unique view on the brain complementary to DTI and T1.  

