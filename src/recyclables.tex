%% This is file `elsarticle-template-1-num.tex',
%%
%% Copyright 2009 Elsevier Ltd
%%
%% This file is part of the 'Elsarticle Bundle'.
%% ---------------------------------------------
%%
%% It may be distributed under the conditions of the LaTeX Project Public
%% License, either version 1.2 of this license or (at your option) any
%% later version.  The latest version of this license is in
%%    http://www.latex-project.org/lppl.txt
%% and version 1.2 or later is part of all distributions of LaTeX
%% version 1999/12/01 or later.
%%
%% The list of all files belonging to the 'Elsarticle Bundle' is
%% given in the file `manifest.txt'.
%%
%% Template article for Elsevier's document class `elsarticle'
%% with numbered style bibliographic references
%%
%% $Id: elsarticle-template-1-num.tex 149 2009-10-08 05:01:15Z rishi $
%% $URL: http://lenova.river-valley.com/svn/elsbst/trunk/elsarticle-template-1-num.tex $
%%

%\documentclass[preprint,authoryear,review,12pt]{elsarticle}
\documentclass[preprint,1p,times]{elsarticle}

%% Use the option review to obtain double line spacing
%% \documentclass[preprint,review,12pt]{elsarticle}

%% Use the options 1p,two column; 3p; 3p,twocolumn; 5p; or 5p,twocolumn
%% for a journal layout:
%% \documentclass[final,1p,times]{elsarticle}
%% \documentclass[final,1p,times,twocolumn]{elsarticle}
%% \documentclass[final,3p,times]{elsarticle}
%% \documentclass[final,3p,times,twocolumn]{elsarticle}
%% \documentclass[final,5p,times]{elsarticle}
%% \documentclass[final,5p,times,twocolumn]{elsarticle}


\usepackage{color}
\usepackage{multirow,booktabs,ctable,array}
\usepackage{lscape}
\usepackage{amsmath}
\usepackage{lineno}
\usepackage{ulem}
\usepackage{setspace}
\usepackage{listings}
\usepackage{float}
\usepackage{listings}
\usepackage{color}
\usepackage{rccol}
\usepackage{hyperref}
\usepackage[table]{xcolor}

    \definecolor{listcomment}{rgb}{0.0,0.5,0.0}
    \definecolor{listkeyword}{rgb}{0.0,0.0,0.5}
    \definecolor{listnumbers}{gray}{0.65}
    \definecolor{listlightgray}{gray}{0.955}
    \definecolor{listwhite}{gray}{1.0}

\newcommand{\lstsetcpplong}
{
\lstset{frame = tb,
        framerule = 0.25pt,
        float,
        fontadjust,
        backgroundcolor={\color{listlightgray}},
        basicstyle = {\ttfamily\scriptsize},
        keywordstyle = {\ttfamily\color{listkeyword}\textbf},
        identifierstyle = {\ttfamily},
        commentstyle = {\ttfamily\color{listcomment}\textit},
        stringstyle = {\ttfamily},
        showstringspaces = false,
        showtabs = false,
        numbers = none,
        numbersep = 6pt,
        numberstyle={\ttfamily\color{listnumbers}},
        tabsize = 2,
        language=,
        floatplacement=!h,
        caption={\small \baselineskip 12pt DiReCT long command line menu which is invoked using the `{\ttfamily {-}{-}help}' option.  The short command line menu is obtained by typing `{\ttfamily {-}h}'},
        captionpos=b,
        label=listing:long
        }
}

\floatstyle{plain}
\newfloat{command}{thp}{lop}
\floatname{command}{Command}

%\usepackage[nomarkers,notablist]{endfloat}

%% if you use PostScript figures in your article
%% use the graphics package for simple commands
%% \usepackage{graphics}
%% or use the graphicx package for more complicated commands
%% \usepackage{graphicx}
%% or use the epsfig package if you prefer to use the old commands
%% \usepackage{epsfig}

%% The amssymb package provides various useful mathematical symbols
\usepackage{amssymb}
%% The amsthm package provides extended theorem environments
% \usepackage{amsthm}
 
 \usepackage{makecell}

%% The lineno packages adds line numbers. Start line numbering with
%% \begin{linenumbers}, end it with \end{linenumbers}. Or switch it on
%% for the whole article with \linenumbers after \end{frontmatter}.
%% \usepackage{lineno}

%% natbib.sty is loaded by default. However, natbib options can be
%% provided with \biboptions{...} command. Following options are
%% valid:

%%   round  -  round parentheses are used (default)
%%   square -  square brackets are used   [option]
%%   curly  -  curly braces are used      {option}
%%   angle  -  angle brackets are used    <option>
%%   semicolon  -  multiple citations separated by semi-colon
%%   colon  - same as semicolon, an earlier confusion
%%   comma  -  separated by comma
%%   numbers-  selects numerical citations
%%   super  -  numerical citations as superscripts
%%   sort   -  sorts multiple citations according to order in ref. list
%%   sort&compress   -  like sort, but also compresses numerical citations
%%   compress - compresses without sorting
%%
%% \biboptions{comma,round}

% \biboptions{}

\providecommand{\OO}[1]{\operatorname{O}\bigl(#1\bigr)}

\graphicspath{
             {./Figures/}
             {../SPIE2013/KK/IXI_results/} 
             }

\long\def\symbolfootnote[#1]#2{\begingroup%
\def\thefootnote{\fnsymbol{footnote}}\footnote[#1]{#2}\endgroup}



\journal{NeuroImage}

\begin{document}


\begin{frontmatter}

\title{Functional Image Processing with Advanced Normalization Tools}

\author[label1]{Brian B.~Avants$^\dagger$}
\author[label2]{Nicholas J.~Tustison$^\dagger$
  \fnref{label0}}
  \fntext[label0]{$\dagger$The first two authors contributed equally to this work.}
\author[label1]{Philip A.~Cook}
\author[label1]{Gang Song}
\author[label1]{Jeffrey R.~Duda}
\address[label1]{Penn Image Computing and Science Laboratory, University of Pennsylvania, Philadelphia, PA}
\address[label2]{Department of Radiology and Medical Imaging, University of Virginia, Charlottesville, VA}

\begin{abstract} 
Advanced Normalization Tools (ANTs), combined with R (world's leading
open source statistics software), is appropriate for scanner-to-study
large-scale multiple modality medical image processing.  Here we
summarize a pipeline for using structural T1 imaging data to guide
processing for arterial spin labeling (ASL), resting state functional
MRI (rsfMRI), diffusion tensor MRI (DTI), cortical thickness and
neuroanatomical labeling of major gyri and deep gray matter structures
(thalamus, hippocampi, caudate, putamen, etc.)  We also summarize how
ANTs output may be fed into R to find interpretable statistical models
of population or individual subject data.
\end{abstract}

\begin{keyword}
advanced normalization tools \sep cortical thickness \sep open science
%% keywords here, in the form: keyword \sep keyword
\end{keyword}

\end{frontmatter}
%
%
\newpage
\section{Neuroimaging in Mild TBI}
Traumatic brain injury (TBI) or mild TBI (mTBI) is a risk in both athletics and the
field of duty
\cite{Toblin2012,Peskind2013,Small2013,Kontos2013,Gosselin2012,Zuckerman2012}.
It is identified as the ``signature injury among United States
military personnel involved in combat in Iraq and Afghanistan''
\cite{McCrea2008,Rigg2011}.  mTBI symptoms can be life-altering and
include sleep disturbance \cite{Mysliwiec2013}, difficulties with
working memory \cite{Bryer2013}, increased reaction times
\cite{Luethcke2011}, post-traumatic stress disorder (PTSD)
\cite{Toblin2012,Bazarian2013} and may
lead to long-term psychiatric disorders such as depression
\cite{Mayer2011,Rapp2012,Toblin2012} or a tendency toward newly adopted addictive
behaviors \cite{Miller2013}.  A recent study reveals that up
to 88\% of soldiers returning from combat may have difficulties with sleep
\cite{Mysliwiec2013}.  % search: mild traumatic brain injury soldier
mTBI remains a clinical diagnosis \cite{MacDonald2011} based on self-report measurements
because objective measurements of mTBI remain an active research area \cite{Rapp2012}.
While invasive TBI biomarkers such as PET (requiring injection with a
radioactive isotope) \cite{Small2013} or cerebrospinal fluid
(requiring a spinal tap) have great value \cite{Zetterberg2013}, magnetic resonance imaging
is a powerful and non-invasive instrument that is widely available and captures more direct
information about brain structure/function and the level of injury
\cite{Fox2013}.  Several studies establish advantages of structural
and functional MRI \cite{Ross2013,Yuh2013,Mayer2011,Morey2012}.
T1 MR image processing improves sensitivity for
atrophy detection in mTBI when compared to standard radiological readings \cite{Ross2013}.  Yuh et al \cite{Yuh2013}
estimate that brain imaging doubles the ability to predict cognitive
outcome measurements at three months post-injury.  Diffusion tensor
imaging reveals consistent neuroimaging findings in animals that
relate to repeated exposure to mild TBI and correlate with pathology
\cite{Bennett2012}.  mTBI correlates in DTI are also present in veterans
\cite{Morey2012,Jorge2012}.  Functional MRI (fMRI) is uniquely able to detect alterations in
both resting and task-adaptive behavior of the brain post-TBI \cite{Roy2010,Mayer2011,Scheibel2012,Stevens2012}.
In summary, multiple MRI modalities are promising biomarkers that 
capture the complex objective sequelae of mTBI that lead to cognitive and behavioral consequences
impairing both quality of life and response to the call of duty.
These objective measurements are therefore valuable in both predicting
outcome as well as in assessing the efficacy of intervention
strategies \cite{Roy2010,Rigg2011}.
% \cite{MacDonald2013,Huang2013,Bennett2012,Scheibel2012,Simmons2012,Rubovitch2011,MacDonald2011,Risdall2011,Sponheim2011,Roy2010,Maruta2010,DeLellis2009,Rosen2009,Huang2009,Moore2009,Salazar2000}.


\section{Image Processing Methods}

All MRI volumes will be processed using Advanced Normalization Tools
(ANTs) \href{http://www.picsl.upenn.edu/ANTS}{(click here for
link)}\cite{Avants2011a,Avants2011,Murphy2011,Tustison2010,Tustison2011,Tustison2011a}
via previously reported procedures validated as highly accurate
\cite{Klein2009,Klein2010,Murphy2011,Avants2011}.  In general, ANTs-based
processing for multiple modalities is based on, first, defining a
template to guide the population normalization and segmentation
procedures \cite{Avants2011a,Avants2011}.  Traditionally, this
approach was used for single modalities but, more recently, is
extended for multiple modalities via multiple-modality cohort-specific templates
that capture the average shape and appearance of T1, DTI and
functional images, as in \cite{Kim2010,Avants2011a,Jain2012,Tustison2012}.  We use the
template to guide brain extraction for all modalities and subsequent
tissue or neuroanatomical parcellation \cite{Avants2011a,Wang2013}.  

Our prior work on TBI reveals effects in thalamus \cite{Avants2008a}
and thalamocortical networks \cite{J.T.Duda2008}.  We used ANTs
methods to uncover widespread structural effects cross-sectionally
\cite{Kim2008} and (more preliminarily) longitudinally \cite{Kim2013}.
In terms of function, we identified TBI effects in default mode
network \cite{Duda2013} and cerebral blood flow \cite{Kim2010}.  Thus,
multiple modality procedures enable several complementary measurements
to characterize TBI:

\noindent{\em Jacobian-based Volume from T1-weighted MRI:} Image
registration methods are able to use T1 MRI to obtain voxel-wise
measurement of the subject structural volume relative to a group local
template or relative to a baseline timepoint.  These measurements are
used in Tensor-Based Morphometry (TBM) which, when performed
symmetrically \cite{Yushkevich2010a,Das2012}, is highly accurate and
provides detailed information on shape change over time
\cite{Brambati2007,Hua2011,Kim2013} or across a population
\cite{Kim2008,Massimo2009,Morgan2011,Hanson2010,Hanson2012}.  

\noindent{\em Cortical thickness from T1-weighted MRI:}  A measurement
that is both sensitive and specific in neurodegeneration
\cite{Stricker2012,Libon2012,McMillan2013} and which is associated
with cognitive decline \cite{Dickerson2011,Avants2010}.  Cortical
thickness has yet to be applied to mTBI but we believe that it will
explain variance in recovery potential within the mTBI population in
particular in the anterior efferent projections of thalamocortical networks
such as the middle frontal and orbitofrontal cortex.

\noindent{\em Measurements derived from DTI:} Fractional anisotropy
measurement of white matter integrity that reveal connectivity
components of neurodegeneration \cite{Zhang2009}, aging
\cite{Kochunov2011} potentially captures a different aspect of the
disease process \cite{Englund2004} and may differentiate tau and
TDP-43 effects \cite{McMillan2013b}.  Relatedly, mean, radial or axial
diffusion relate to age and cognition
\cite{Wu2011,Pal2011,Bava2011,ODwyer2012,Bjoernebekk2012},
neurodegeneration \cite{Salami2012,Whitwell2010} and are particularly
sensitive to TBI \cite{Huang2009,Fox2013}.  We will characterize mean and radial diffusion
within thalamocortical networks.

\noindent{\em BOLD or ASL Network Analysis:} BOLD-based network analysis
\cite{Spoormaker2010,Sanz-Arigita2010} has emerged as a powerful tool
with specific value in both brain injury
\cite{Mayer2011,Scheibel2012,Zhou2013}, the study of intervention
strategies \cite{Roy2010,FeldsteinEwing2011} and measurement of pain
\cite{Mayhew2013}.  ASL may also be employed in this manner
\cite{Jann2013} and has been processed with ANTs tools and canonical
correlation analysis \cite{Duda2013}.  We propose network measures
based on constant graph density across subjects
which controls for inter-subject variation in base correlation levels
\cite{Liu2008, Power2011, Schwarz2011, Braun2012, Liang2012}. 
On the binarized graphs, we compute betweenness, small-worldness and
related metrics. 

\noindent{\em Cerebral blood flow (CBF) from ASL:}  This functional
quantitative measure (versus the relative values provided by BOLD) has
the potential to reveal alterations in the brain due to injury
\cite{Kim2008}, pain \cite{Howard2011}, pharmacological intervention \cite{Black2010,Jenkins2012} or that precede visible structural change and may
indicate cortical reorganization \cite{Hayward2010}.  ASL-based CBF
also accurately recapitulates PET imaging in Alzheimer's disease \cite{Chen2011,Mak2012}.
CBF is a more repeatable functional measurement than BOLD
\cite{Liu2007,Aguirre2012}, may be used in network analysis in lieu of
or combination with 
BOLD \cite{Duda2013} and provides a unique view on the brain complementary to DTI and T1.  



\section{Statistical Methods}\label{sec:R}
R~is a popular
programming/scripting language that is designed to make advanced
statistical analysis accessible.  Because of its ease of use, large user base, and
extensive community development, it is increasingly used by
researchers, statisticians, and data analysts from fields as varied as
economics, machine learning, pharmaceuticals, and finance. When
combined with the image processing utilities available in ANTs, R
provides a convenient and powerful interface for performing common
statistical analyses of imaging data.  The basic form of a statistical model in R is 
\begin{equation}
\text{Outcome} \approx \text{Predictor 1} + \text{Predictor 2} + \text{Predictor 3} + ...
\label{eqn:r_syntax}
\end{equation}
Factor and continuous variable predictors can be combined seamlessly,
and a wide variety of model types, including linear models with
Gaussian noise, logistic, and Poisson models are available.  To
perform image analysis of the relationship between imaging data and
cognition,  one may use a model such as:
\begin{equation}
\text{cognition} \approx \text{image-value} + \text{other demographic variables}. 
\label{eqn:ROI}
\end{equation}
Given R's unified interface, conducting voxel-wise morphometric
studies of cortical thickness or fractional anisotropy (FA) are
equally easy.  Similarly network or CBF measurements may be employed
as image-based predictors.  ANT with R also enables more sophisticated functional studies.  ANTs provides utilities for all necessary preprocessing of time-series data, including motion correction, detrending, and noise reduction (CompCor).  Due to the ease of using factorial predictors in R, task-based fMRI images can be described using the same syntax: 
\begin{equation}
\text{Preprocessed fMRI data} \approx \text{Presence of task} + \text{nuisance regressors}.
\label{eqn:fmri}
\end{equation}
The flexibility of R's linear model functionality also makes tasks
that are not typically thought of as regression problems easy to
perform. Finally, mixed effects models are readily available in R and
may allow more powerful longitudinal or other repeated measures modeling
to be performed. 
\section*{Acknowledgments}

\section*{References}

\bibliographystyle{elsarticle-harv}
\bibliography{src/references}


%% Authors are advised to submit their bibtex database files. They are
%% requested to list a bibtex style file in the manuscript if they do
%% not want to use model1-num-names.bst.

%% References without bibTeX database:

% \begin{thebibliography}{00}

%% \bibitem must have the following form:
%%   \bibitem{key}...
%%

% \bibitem{}

% \end{thebibliography}


\end{document}

%%
%% End of file `elsarticle-template-1-num.tex'.
