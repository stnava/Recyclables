
\section{Statistical Methods}\label{sec:R}
R~is a popular
programming/scripting language that is designed to make advanced
statistical analysis accessible.  Because of its ease of use, large user base, and
extensive community development, it is increasingly used by
researchers, statisticians, and data analysts from fields as varied as
economics, machine learning, pharmaceuticals, and finance. When
combined with the image processing utilities available in ANTs, R
provides a convenient and powerful interface for performing common
statistical analyses of imaging data.  The basic form of a statistical model in R is 
\begin{equation}
\text{Outcome} \approx \text{Predictor 1} + \text{Predictor 2} + \text{Predictor 3} + ...
\label{eqn:r_syntax}
\end{equation}
Factor and continuous variable predictors can be combined seamlessly,
and a wide variety of model types, including linear models with
Gaussian noise, logistic, and Poisson models are available.  To
perform image analysis of the relationship between imaging data and
cognition,  one may use a model such as:
\begin{equation}
\text{cognition} \approx \text{image-value} + \text{other demographic variables}. 
\label{eqn:ROI}
\end{equation}
Given R's unified interface, conducting voxel-wise morphometric
studies of cortical thickness or fractional anisotropy (FA) are
equally easy.  Similarly network or CBF measurements may be employed
as image-based predictors.  ANT with R also enables more sophisticated functional studies.  ANTs provides utilities for all necessary preprocessing of time-series data, including motion correction, detrending, and noise reduction (CompCor).  Due to the ease of using factorial predictors in R, task-based fMRI images can be described using the same syntax: 
\begin{equation}
\text{Preprocessed fMRI data} \approx \text{Presence of task} + \text{nuisance regressors}.
\label{eqn:fmri}
\end{equation}
The flexibility of R's linear model functionality also makes tasks
that are not typically thought of as regression problems easy to
perform. Finally, mixed effects models are readily available in R and
may allow more powerful longitudinal or other repeated measures modeling
to be performed. 