
\section{Neuroimaging in Mild TBI}
Traumatic brain injury (TBI) or mild TBI (mTBI) is a risk in both athletics and the
field of duty
\cite{Toblin2012,Peskind2013,Small2013,Kontos2013,Gosselin2012,Zuckerman2012}.
It is identified as the ``signature injury among United States
military personnel involved in combat in Iraq and Afghanistan''
\cite{McCrea2008,Rigg2011}.  
Repeated exposure to mTBI makes recovery increasingly difficult and
leads to long-term susceptibility to recurring injuries
\cite{Shively2012}.  % [NEED BETTER REFERENCE]
mTBI symptoms can be life-altering and
include sleep disturbance \cite{Mysliwiec2013}, difficulties with
working memory \cite{Bryer2013}, increased reaction times
\cite{Luethcke2011}, post-traumatic stress disorder (PTSD)
\cite{Toblin2012,Bazarian2013} and may
lead to long-term psychiatric disorders such as depression
\cite{Mayer2011,Rapp2012,Toblin2012} or a tendency toward newly adopted addictive
behaviors \cite{Miller2013}.  A recent study reveals that up
to 88\% of soldiers returning from combat may have difficulties with sleep
\cite{Mysliwiec2013}.  % search: mild traumatic brain injury soldier
mTBI remains a clinical diagnosis \cite{MacDonald2011} based on self-report measurements
because objective measurements of mTBI remain an active research area \cite{Rapp2012}.
While invasive TBI biomarkers such as PET (requiring injection with a
radioactive isotope) \cite{Small2013} or cerebrospinal fluid
(requiring a spinal tap) have great value \cite{Zetterberg2013}, magnetic resonance imaging
is a powerful and non-invasive instrument that is widely available and captures more direct
information about brain structure/function and the level of injury
\cite{Fox2013}.  Several studies establish advantages of structural
and functional MRI \cite{Ross2013,Yuh2013,Mayer2011,Morey2012}.
T1 MR image processing improves sensitivity for
atrophy detection in mTBI when compared to standard radiological readings \cite{Ross2013}.  Yuh et al \cite{Yuh2013}
estimate that brain imaging doubles the ability to predict cognitive
outcome measurements at three months post-injury.  Diffusion tensor
imaging reveals consistent neuroimaging findings in animals that
relate to repeated exposure to mild TBI and correlate with pathology
\cite{Bennett2012}.  mTBI correlates in DTI are present in veterans
\cite{Morey2012,Jorge2012} and improve longitudinal outcome prediction
at a patient-specific level \cite{Sidaros2008}.  Functional MRI (fMRI) is uniquely able to detect alterations in
both resting and task-adaptive behavior of the brain post-TBI
\cite{Roy2010,Mayer2011,Scheibel2012,Stevens2012}.  Both DTI and fMRI,
including arterial spin labeling (ASL) for perfusion quantification
\cite{Ge2009,Grossman2013}, reveal mTBI effects that can elude 
standard clinical assessments including T1 or T2 MRI.
In summary, multiple MRI modalities are promising biomarkers that 
capture the complex objective sequelae of mTBI that lead to cognitive and behavioral consequences
impairing both quality of life and response to the call of duty.
These objective measurements are therefore valuable in both predicting
outcome and assessing interventional or therapeutic strategies \cite{Roy2010,Rigg2011,Niemeier2011}.
% \cite{MacDonald2013,Huang2013,Bennett2012,Scheibel2012,Simmons2012,Rubovitch2011,MacDonald2011,Risdall2011,Sponheim2011,Roy2010,Maruta2010,DeLellis2009,Rosen2009,Huang2009,Moore2009,Salazar2000}.

